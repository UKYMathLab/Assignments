 % Gemini theme
% https://github.com/anishathalye/gemini

\documentclass[final]{beamer}

% Packages
% ====================

\usepackage[T1]{fontenc}
\usepackage{lmodern}
\usepackage{comment}
\usepackage[size=custom,width=120,height=72,scale=1.0]{beamerposter}
\usetheme{gemini}
\usecolortheme{gemini}
\usepackage{graphicx}
\usepackage{booktabs}
\usepackage{tikz}
\usepackage{pgfplots}
\usepackage{amsmath,amsthm,amssymb,amsfonts}
\usepackage{mathrsfs}
\usepackage{tabularx}
\usepackage{color, soul}
\usepackage{todonotes}
\usepackage{caption}
%\usepackage[outercaption]{sidecap} 
\usepackage{amsmath, amssymb, amsthm, wasysym,todonotes,bm}
\usepackage[margin=3cm]{geometry}

% just kt things
\newcommand\inner[2]{\langle #1, #2 \rangle}
\newcommand\complexconj[1]{ \overline{#1}}
\newcommand{\N}{\mathbb{N}}
\newcommand{\Z}{\mathbb{Z}}
\newcommand{\Q}{\mathbb{Q}}
\newcommand{\R}{\mathbb{R}}
%\newcommand{\C}{\mathbb{C}}
\newcommand{\Sp}{\text{Sp}}
\newcommand{\Mod}{\text{Mod}}




% ====================
% Lengths
% ====================

% If you have N columns, choose \sepwidth and \colwidth such that
% (N+1)*\sepwidth + N*\colwidth = \paperwidth
\newlength{\sepwidth}
\newlength{\colwidth}
\setlength{\sepwidth}{0.025\paperwidth}
\setlength{\colwidth}{0.3\paperwidth}

\newcommand{\separatorcolumn}{\begin{column}{\sepwidth}\end{column}}

%\definecolor{forestgreen}{rgb}{0.0, 0.27, 0.13}
%\setbeamercolor{block title}{bg=blue,fg=white}

% ====================
% Title
% ====================

\title{On Congruence Subgroups of the Braid Group}

\author{Jessica Appel\inst{1}, Katie Gravel\inst{2}, Annie Holden\inst{3}}

\institute[shortinst]{\inst{1} University of Kentucky \samelineand \inst{2} Massachusetts Institute of Technology \samelineand \inst{3} Colby College}

% ====================
% Body
% ====================

\begin{document}

\begin{frame}[t]
\begin{columns}[t]
\separatorcolumn

\begin{column}{\colwidth}

%\iffalse
 \begin{block}{\begin{huge}
    Braids
  \end{huge}} 
  
\begin{huge}
\vspace*{5mm} \emph{Goal:} Understand the structure of congruence subgroups of the braid group.
\end{huge}
  \end{block}
 %\fi
 
\begin{figure}
    \includegraphics[width=37.5cm]{braid_relation_final.pdf}
    %\captionsetup{labelformat = empty}
    %\caption{\begin{huge} Braid Relation \end{huge}}
\end{figure} 
%\begin{figure}
 %   \includegraphics[height=3in]{PureBraid1.pdf}
  %  \caption{}
%\end{figure} 
%\vspace*{-55mm}

\vspace{20mm}
\begin{figure}
    \includegraphics[width=35cm]{pure_braids_final.pdf}
\end{figure} 


\begin{figure}
    \includegraphics[width=25cm]{level_four_braid_final.pdf}
\end{figure} 

\end{column}

\separatorcolumn


\begin{column}{\colwidth} 
\begin{block}{\begin{huge}
    Integral Burau Representation
  \end{huge}} 

\begin{huge} \vspace*{-25mm}
\begin{align*}
    \rho_{-1}: B_n &\rightarrow GL(n, \Z)\\
    \sigma_i &\mapsto I_{i-1} \oplus \begin{bmatrix}
2 & -1 \\
1 & 0 
\end{bmatrix} \oplus I_{n-i-1}
\end{align*}
\vspace*{-5mm}
$r_N$ is the usual mod $N$ reduction map
\vspace*{3mm}
$$B_n[N]=ker(r_N \circ \rho_{-1})$$
\end{huge}
  \end{block}

\begin{block}
  {\begin{huge}
    \vspace*{5mm} Problem I: Generating Sets
  \end{huge}} \begin{huge} 
  
 \vspace*{5mm} \emph{Question}: What is a natural generating set for $B_n[4]$? How big is it?
\end{huge} 

\begin{huge}

Margalit and Kordek: Size lower bounded by 
  $${n\choose 2}+3{n\choose 3}+3{n\choose 4}\sim \text{O}(n^4)$$
 
 \end{huge}


\begin{huge}

Schreier's method $\rightsquigarrow$ exponential generating set
 
Use recurrence relation to reduce generating set


\vspace*{-1mm}
{\bf {\begin{huge} Theorem. \end{huge}}}  \begin{Huge}
$${ \text{\# generators of} \; B_n[4]\sim \text{O}(n^5)}$$
\end{Huge}
  


\end{huge}
\end{block} 

\begin{block}{\begin{huge} Acknowledgements \end{huge}}

\begin{huge}
 We would like to thank Dr.\@ Wade Bloomquist, Dr. Dan Margalit, Georgia Tech, and the National Science Foundation.
 \end{huge}
  \end{block}
\end{column}


  \separatorcolumn

\begin{column}{\colwidth}

 \vspace*{-3mm} \begin{block}{
  \begin{huge}
    Problem II: \bm{$PB_n^\ell$} and \bm{$B_n[2\ell]$}
  \end{huge}} 

\begin{huge} \vspace*{5mm}  \emph{Question:} What is the relationship between $PB_n^\ell$ and $B_n[2\ell]$ for varying $\ell$?
\end{huge}

\vspace*{-10mm} 
\begin{huge} \begin{itemize}
\item[]Brendle and Margalit: $PB_n^2 = B_n[4]$
\end{itemize} 

\vspace*{-7mm}
{ \bf{\begin{huge} Theorem. \end{huge}}}

\vspace*{-7mm}
\begin{itemize} \begin{huge}
  \item[] For $\ell = 2^k$, $PB_n^\ell \subset B_n[2\ell]$
  
\item[] For $\ell = 6,10,12$ or $\ell$ odd, $PB_n^\ell \not\subset B_n[2\ell]$
\end{huge}
\end{itemize}
\vspace*{-7mm}

{\bf{\begin{huge} Conjecture. \end{huge}}}
\vspace*{-7mm}
\begin{huge}
\begin{itemize}
  
\item[] $\; \ell = 2^k \iff PB_n^\ell \subset B_n[2\ell]$

\end{itemize}
\end{huge}

\end{huge}

\end{block}

  \begin{block}{
  \begin{huge}
  \vspace*{5mm} 
    Problem III:   Quotients
  \end{huge}} 
 
\begin{huge} \vspace*{5mm}  \emph{Question:} What can we say about quotients of Burau levels? \end{huge}

\vspace*{-7mm}
\begin{huge}
  \begin{itemize}
  \item[] Artin: $B_n / PB_n \cong S_n$
  
  \item[] Stylianakis: $B_n[p] / B_n[2p] \cong S_n$ for $p$ prime 
  \end{itemize}
    \end{huge} \vspace*{-7mm}

{\bf{ \begin{huge} Theorem. \end{huge}} }
\begin{huge}
\vspace*{-7mm}
\begin{itemize} 
\item[] $B_n[\ell] / B_n[2\ell] \cong S_n$ for odd $\ell$

        
\item[] $B_n[\ell] / B_n[2\ell] \cong (\Z /2\Z)^{{n\choose 2}}$ for even $\ell$
  
\end{itemize}
\end{huge}

  \end{block} 
\begin{Large}
%\vspace*{-15mm}

 



%  \begin{block}{References}
%\begin{thebibliography}{}
%\bibitem{}
%  Brendle, Tara and Margalit, Dan. The Level Four Braid Group. 2014. %https://arxiv.org/abs/1410.7416.
  
%\bibitem{}
%  Karoske, Yvonne. Reidemeister-Schreier Procedure using Grobner Bases Techniques. 2013. 
  %http://cocoa.dima.unige.it/conference/cocoa2013/posters/YvonneKaroske.pdf.
  
%\bibitem{}
%  Stylianakis, Charalampos. Congruence Subgroups of the Braid Group. 2016. %https://arxiv.org/abs/1609.05673.  
 % \end{thebibliography}


    %\nocite{*}
    %\footnotesize{\bibliographystyle{plain}\bibliography{poster}}
   
  %\end{block}
  
\end{Large}


\begin{figure}
    \includegraphics[height=2.5in]{schoolofmathlogo.png}
\hspace{1cm}
    \includegraphics[height=2.5in]{NSFLogo.png}
\end{figure} 

\end{column}

\separatorcolumn
\end{columns}
\end{frame}

\end{document}
